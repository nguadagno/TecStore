\documentclass[12pt,a4paper]{article}
\usepackage[utf8]{inputenc}
\usepackage{amsmath}
\usepackage{amsfonts}
\usepackage{amssymb}
\usepackage{graphicx}
\graphicspath{ {./img/} }
\usepackage{hyperref}
\usepackage{array}
\usepackage[table]{xcolor}
\usepackage{xcolor,colortbl}
\usepackage{multirow}
\usepackage[a4paper, total={6in, 8in}]{geometry}

\usepackage{titlesec}

\setcounter{secnumdepth}{4}

\titleformat{\paragraph}
{\normalfont\normalsize\bfseries}{\theparagraph}{1em}{}
\titlespacing*{\paragraph}
{0pt}{3.25ex plus 1ex minus .2ex}{1.5ex plus .2ex}

\author{Natale Guadagno, Paolo Patrone}
\title{Requisites Analysis Document - TecStore}
\renewcommand{\contentsname}{Contenuti}

\usepackage{hyperref}
\hypersetup{
    colorlinks,
        citecolor=blue,
    filecolor=blue,
    linkcolor=blue,
    urlcolor=blue,
    linktocpage
}

\begin{document}

\maketitle
\newpage
\tableofcontents
\newpage
\newgeometry{top=0.5in,bottom=0.5in,right=0.5in,left=0.5in}
\section*{Partecipanti}
\begin{center}
\begin{tabular} {|c|c|}
\hline
\textbf{Nome} & \textbf{Matricola} \\
\hline
Guadagno Natale & 0512106546 \\
Patrone Paolo & 0512106153 \\
\hline
\end{tabular}
\end{center}
\newpage

\section{Introduzione}
\subsection{Scopo del sistema}
Il sistema si propone come interfaccia unificata e semplificata per la gestione di una realtà complessa come un e-commerce.

Le interfacce sono quindi pensate per essere di immediata lettura e accessibili anche per chi ha poca dimestichezza con sistemi informatici.

\subsection{Obiettivi di design}
Per garantire un livello di accessibilità universale sono previsti più test di usabilità per ogni interfaccia, in modo da evidenziare criticità risolvibili. \\
Per facilitare l'utilizzo della piattaforma, ci si è posto anche l'obiettivo di avere un'interfaccia molto reattiva con tempi di risposta molto brevi e molti messaggi di conferma per assicurare gli utenti che le loro operazioni sono state effettuate.

\subsubsection{Criteri prestazionali}
\begin{tabular}{|p{4cm}|p{12cm}|}
\hline
\textbf{Tempi di risposta} & Il sistema si prepone l'obiettivo di essere il più possibile reattivo, ovvero di effettuare la maggioranza delle operazioni semplici come autenticazione, registrazione, risposta ad un ticket in meno di 1s e al più 10s per operazioni più complesse come la ricerca degli articoli. \\
\hline
\textbf{Throughput} & Il sistema si prepone l'obiettivo di gestire anche picchi improvvisi di utenza senza grossi rallentamenti. Sono previsti più webserver con \emph{load balancer} che permettono quindi di gestire molti più utenti. \\
\hline
\textbf{Memorizzazione di dati} & Il sistema utilizzerà un database MySQL per la memorizzazione di dati testuali (informazioni degli utenti, lista degli articoli, ...) e, per evitare di rendere i file del database troppo grandi, i file immagine saranno memorizzati su disco. Tutti questi dati riceveranno dei backup periodici con strategia 3-2-1, ovvero 3 copie dei dati, su 2 dispositivi fisici diversi e almeno 1 copia in un'altra posizione geografica. \\
\hline
\end{tabular}

\subsubsection{Criteri di affidabilità}
\begin{tabular}{|p{4cm}|p{12cm}|}
\hline
\textbf{Robustezza} & L'hardware scelto per il sistema deve essere di livello aziendale e resistente a eventuali problemi hardware come la rottura di un disco fisso, attraverso l'uso di tecnologia RAID, di un alimentatore, con alimentatori ridondanti, alla mancanza di corrente attraverso batterie e sistemi UPS. In nessun caso un singolo crash hardware deve compromettere l'accessibilità al sito.
In più, ci si proteggerà da problematiche software usando versioni del webserver e del sistema operativo testate e senza problemi noti. \\
\hline
\t1extbf{Disponibilità} & Il sistema deve garantire un \emph{uptime} (tempo di attività) di almeno il 99.9\%, ovvero un \textit{downtime} (tempo di inattività) annualizzato di meno di 9 ore. Ciò è cruciale per far sì che l'utenza non venga scoraggiata dall'utilizzo di TecStore come negozio primario, creando perdite potenziali molto alte, soprattutto nei periodi di maggior afflusso di utenza. \\
\hline
\textbf{Tolleranza agli errori} & Il sistema deve garantire l'accessibilità anche in condizioni non ottimali, come il crash di uno dei webserver o un blackout. Ciò è garantito dai componenti ridondanti e dai backup. \\
\hline
\textbf{Sicurezza} & Il sistema deve prevedere tutte le pratiche di sicurezza fondamentali, come l'utilizzo di SSL per la trasmissione dei dati, l'utilizzo di \textit{hashing} e \textit{salt} per le password memorizzate nel database, tutti i dati delle carte di credito e anagrafiche devono essere cifrati prima di essere inseriti nel database utilizzando una cifratura robusta con una chiave che non deve essere esposta pubblicamente per nessun motivo.
In caso di tentativo di accesso a schermate riservate da parte di un utente consumatore o viceversa, ovvero un utente del personale che cerca di accedere al catalogo, deve essere previsto un avviso e un \textit{redirect} ad una pagina correttamente accessibile da quel tipo di utente. \\
\hline
\end{tabular}

\subsubsection{Criteri di manutenzione}
Lo sviluppo del sistema sarà condotto in modo da facilitare l'estensione utilizzando linguaggi e tecnologie standard come HTML5, CSS3, Bootstrap e Java. Il codice deve essere quindi scritto in modo che sia facile intervenire, sia per risolvere eventuali bug, sia per aggiungere nuove funzionalità.

\subsection{Definizioni, acronimi e abbreviazioni}
\begin{itemize}
\item TecStore: nome della piattaforma 
\item Cliente: utente che può acquistare, vendere, richiedere assistenza
\item Centralinista: utente che controlla le vendite e fornisce assistenza ai clienti
\item Magazziniere: utente che controlla la spedizione degli ordini
\item Amministratore catalogo: utente che gestisce le vendite da parte della piattaforma
\item Amministratore personale: utente che gestisce gli account degli altri utenti, fatta eccezione per i clienti
\item DBMS: Database Management System, sistema di gestione di una base di dati
% TODO
\end{itemize}

\subsection{Panoramica}
In questo documento sono descritti in dettaglio:

\begin{itemize}
\item Decomposizione in sottosistemi: in cui viene esposto come il sistema è suddiviso in sottosistemi e come ogni sottosistema interagisce con gli altri.
\item Mapping hardware/software: in cui vengono descritti i requisiti hardware e software su cui il sistema dovrà girare.
\item Gestione dei dati persistenti: in cui viene descritto come i dati verranno memorizzati dal sistema.
\item Controllo degli accessi: in cui vengono descritte le funzionalità messe a disposizione per ogni utente.
\item Condizioni di boundary: in cui verranno descritte le condizioni limite del sistema come avvio, spegnimento, manutenzione e gestione dei fallimenti.
\end{itemize}

\section{Sistema proposto}

\subsection{Panoramica}
L'architettura del sistema è di tipo client/server. Il server resta in attesa di richiesta da parte degli utenti e risponde nel minor tempo possibile.
I motivi per la scelta di un'architettura client/server sono principalmente:
\begin{itemize}
\item Portabilità: il sistema è facilmente portabile ed accessibile da una varietà di dispositivi senza alcuna necessità di modifica.
\item Performance: il sistema deve offrire buone prestazioni CPU e ottime prestazioni I/O per garantire una buona reattività.
\item Scalabilità: il sistema è pensato per essere facilmente scalabile orizzontalmente.
\item Affidabilità: il sistema prevede più ridondanze e backup per garantire l'accessibilità da parte degli utenti.
\end{itemize}

\subsection{Controllo degli accessi}



\newgeometry{top=0.5in,bottom=0.5in,right=0.5in,left=0.1in}
\begin{table}[]
\begin{tabular}{|l|l|l|l|l|l|l|}
\hline
\textbf{} &
  \textbf{Account} &
  \textbf{Assistenza} &
  \textbf{Autenticazione} &
  \textbf{Carrello} &
  \textbf{Ordine} &
  \textbf{Vendita} \\ \hline
\cellcolor[HTML]{C0C0C0}\textbf{\begin{tabular}[c]{@{}l@{}}Utente \\ non autenticato\end{tabular}} &
  {\color[HTML]{34FF34} \begin{tabular}[c]{@{}l@{}}$\checkmark$\\ Solo per\\ registrazione\end{tabular}} &
  {\color[HTML]{FE0000} $\times$} &
  {\color[HTML]{34FF34} $\checkmark$} &
  {\color[HTML]{FE0000} $\times$} &
  {\color[HTML]{FE0000} $\times$} &
  {\color[HTML]{34FF34} \begin{tabular}[c]{@{}l@{}}$\checkmark$\\ Solo per \\ ricerca e \\ visualizzazione \\ dettagli\end{tabular}} \\ \hline
\cellcolor[HTML]{C0C0C0}\textbf{Cliente} &
  {\color[HTML]{34FF34} \begin{tabular}[c]{@{}l@{}}$\checkmark$\\ Solo per\\ modifica\end{tabular}} &
  {\color[HTML]{34FF34} $\checkmark$} &
  {\color[HTML]{FE0000} $\times$} &
  {\color[HTML]{34FF34} $\checkmark$} &
  {\color[HTML]{34FF34} $\checkmark$} &
  {\color[HTML]{34FF34} $\checkmark$} \\ \hline
\cellcolor[HTML]{C0C0C0}\textbf{Centralinista} &
  {\color[HTML]{FE0000} $\times$} &
  {\color[HTML]{34FF34} \begin{tabular}[c]{@{}l@{}}$\checkmark$\\ Solo per \\ risposta\\ a ticket \\ esistenti\end{tabular}} &
  {\color[HTML]{FE0000} $\times$} &
  {\color[HTML]{FE0000} $\times$} &
  {\color[HTML]{FE0000} $\times$} &
  {\color[HTML]{34FF34} \begin{tabular}[c]{@{}l@{}}$\checkmark$\\ Solo per\\ cambiamenti\\ di stato per \\ una vendita\\ ``In attesa"\end{tabular}} \\ \hline
\cellcolor[HTML]{C0C0C0}\textbf{Magazziniere} &
  {\color[HTML]{FE0000} $\times$} &
  {\color[HTML]{FE0000} $\times$} &
  {\color[HTML]{FE0000} $\times$} &
  {\color[HTML]{FE0000} $\times$} &
  {\color[HTML]{34FF34} \begin{tabular}[c]{@{}l@{}}$\checkmark$\\ Solo per\\ cambiamenti\\ di stato per \\ un ordine\\ ``In attesa"\end{tabular}} &
  {\color[HTML]{FE0000} $\times$} \\ \hline
\cellcolor[HTML]{C0C0C0}\textbf{\begin{tabular}[c]{@{}l@{}}Amministratore \\ Catalogo\end{tabular}} &
  {\color[HTML]{FE0000} $\times$} &
  {\color[HTML]{FE0000} $\times$} &
  {\color[HTML]{FE0000} $\times$} &
  {\color[HTML]{FE0000} $\times$} &
  {\color[HTML]{FE0000} $\times$} &
  {\color[HTML]{34FF34} $\checkmark$} \\ \hline
\cellcolor[HTML]{C0C0C0}\textbf{\begin{tabular}[c]{@{}l@{}}Amministratore \\ Personale\end{tabular}} &
  {\color[HTML]{34FF34} $\checkmark$} &
  {\color[HTML]{FE0000} $\times$} &
  {\color[HTML]{FE0000} $\times$} &
  {\color[HTML]{FE0000} $\times$} &
  {\color[HTML]{FE0000} $\times$} &
  {\color[HTML]{FE0000} $\times$} \\ \hline
\end{tabular}
\end{table}

\newgeometry{top=0.5in,bottom=0.5in,right=0.5in,left=0.5in}

\end{document}