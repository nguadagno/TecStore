\documentclass[12pt,a4paper]{article}
\usepackage[utf8]{inputenc}
\usepackage{amsmath}
\usepackage{amsfonts}
\usepackage{amssymb}
\usepackage{graphicx}
\graphicspath{ {./img/} }
\usepackage{hyperref}
\usepackage{array}
\usepackage[table]{xcolor}
\usepackage[a4paper, total={6in, 8in}]{geometry}


\author{Natale Guadagno, Paolo Patrone}
\title{Requisites Analysis Document - TecStore}
\renewcommand{\contentsname}{Contenuti}

\usepackage{hyperref}
\hypersetup{
    colorlinks,
        citecolor=blue,
    filecolor=blue,
    linkcolor=blue,
    urlcolor=blue,
    linktocpage
}

\begin{document}

\maketitle
\newpage
\tableofcontents
\newpage
\newgeometry{top=0.5in,bottom=0.5in,right=0.5in,left=0.5in}
\section*{Partecipanti}
\begin{center}
\begin{tabular} {|c|c|}
\hline
\textbf{Nome} & \textbf{Matricola} \\
\hline
Guadagno Natale & 0512106546 \\
Patrone Paolo & 0512106153 \\
\hline
\end{tabular}
\end{center}


\section*{Revision History}
\begin{center}
\begin{tabular} {|c|c|c|}
\hline
\textbf{Data} & \textbf{Versione} & \textbf{Descrizione} \\
01/12/2021 & 0.1 & Prima stesura \\
\hline

\hline
\end{tabular}
\end{center}

\newpage

\section{Introduzione}

\subsection{Scopo del sistema}
Il sistema si pone l'obiettivo di facilitare le operazioni di gestione e controllo di operazioni di compravendita e di operazioni gestionali che non sono a carico dell'utenza, come la gestione del personale e dell'inventario. Sono previste viste e interfacce distinte per le varie operazioni per semplificare l'utilizzo ed evitare confusione.

\subsection{Ambito del sistema}
TecStore è una piattaforma web che permette ad utenti di comprare e vendere materiale tecnologico, che può essere messo in vendita da altri utenti o dalla piattaforma stessa. Trattandosi di un negozio piccolo, il catalogo è relativamente limitato, ma si punta a creare utenza attraverso prezzi vantaggiosi e un servizio clienti di qualità.
L'utenza della piattaforma si divide in:
\begin{itemize}
\item Clienti, che, una volta registrati ed autenticati, possono acquistare articoli
\item Venditori, che mettono articoli in vendita
\item Centralinisti, che gestiscono i \emph{ticket}
\item Magazzinieri, che gestiscono le spedizioni degli articoli
\item Amministratori catalogo, che gestiscono gli articoli messi in vendita dalla piattaforma
\item Amministratori personale, che gestiscono la presenza nel sistema dei dipendenti
\end{itemize}

\subsection{Obiettivi del progetto}
 TODO

\subsection{Definizioni, acronimi ed abbreviazioni}
 TODO

\section{Sistema proposto}
 TODO
 
\subsection{Identificazione attori}
 TODO
 
\subsection{Requisiti funzionali}
 TODO

\subsection{Requisiti non funzionali}
 TODO
 
\subsection{Scenari}
 TODO

\section{Use case}
\subsection{Clienti}
\subsubsection{Registrazione di un cliente}
\label{UC:1}
\begin{tabular}{|c|p{12cm}|}
\hline
\textbf{ID} & UC1 Registrazione \\
\hline
\textbf{Nome} & Registrazione di un nuovo utente \\
\hline
\textbf{Partecipanti} & Utente non registrato \\
\hline
\textbf{Condizione d'ingresso} & Un nuovo utente visita il sito per la prima volta. \\
\hline
\textbf{Flusso di eventi} &
\begin{minipage}{12cm}
\begin{tabular}{p{5.5cm} p{5.5cm}}
\textbf{Utente} & \textbf{Sistema} \\
Si collega al sito, nella homepage fa click sul tasto "Registrati" & \\
& Mostra all'utente il form per la registrazione in cui inserire tutti i dati personali, incluso nome, cognome, indirizzo, numero di carta di credito. \\
Inserisce tutti i suoi dati, fa click su "Registrati". & \\
& Se le credenziali sono corrette, invia all'utente una mail di conferma e chiede all'utente di utilizzare il link per confermare l'account. \\
\end{tabular}
\end{minipage} \\

\hline
\textbf{Condizione d'uscita} & L'utente si registra. \\

\hline
\end{tabular}

\subsubsection{Autenticazione di un utente}
\label{UC:2}
\begin{tabular}{|c|p{12cm}|}
\hline
\textbf{ID} & UC2 Login \\
\hline
\textbf{Nome} & Login di un utente \\
\hline
\textbf{Partecipanti} & Utente registrato \\
\hline
\textbf{Condizione d'ingresso} & Un cliente della piattaforma visita il sito da un nuovo dispositivo. \\
\hline
\textbf{Flusso di eventi} &
\begin{minipage}{12cm}
\begin{tabular}{p{5.5cm} p{5.5cm}}
\textbf{Utente} & \textbf{Sistema} \\
Si collega al sito, nella homepage fa click sul tasto "Login" & \\
& Mostra all'utente il form per il login, chiedendo email e password. \\
Inserisce tutti i suoi dati, fa click su "Login". & \\
& Se le credenziali sono corrette, reindirizza l'utente alla homepage. \\
\end{tabular}
\end{minipage} \\

\hline
\textbf{Condizione d'uscita} & L'utente effettua il login. \\
\hline
\end{tabular}

\subsubsection{Possibilità di acquisto per utenti registrati}
\label{UC:3}
\begin{tabular}{|c|p{12cm}|}
\hline
\textbf{ID} & UC3 Acquisto \\
\hline
\textbf{Nome} & Possibilità di acquisto per utenti registrati \\
\hline
\textbf{Partecipanti} & Utente registrato, utente magazziniere \\
\hline
\textbf{Condizione d'ingresso} & Un cliente della piattaforma vuole acquistare un articolo. \\
\hline
\textbf{Flusso di eventi} &
\begin{minipage}{12cm}
\begin{tabular}{p{4cm} p{4cm} p{3cm}}
\textbf{Utente} & \textbf{Sistema} & \textbf{Magazziniere}\\
Si collega al sito, nella homepage inserisce il nome dell'articolo desiderato nella barra di ricerca fa click sul tasto ``Cerca". & \\
& Mostra all'utente un elenco di articoli corrispondenti alla parola cercata. \\
Fa click su uno degli articoli. \\
& Reindirizza l'utente ad una pagina con i dettagli dell'articolo. \\
Aggiunge l'articolo al carrello. \\
& Reindirizza l'utente alla pagina del carrello. \\
Fa click sul tasto "Acquista". \\
& Reindirizza l'utente alla pagina di conferma dell'ordine. \\
Fa click sul tasto "Conferma". \\
& Invia una notifica di nuovo ordine ad un magazziniere. \\
& & Riceve la notifica di nuovo ordine, procura l'articolo dal magazzino, stampa la bolla di spedizione, prenota il corriere e conferma la spedizione. \\
& Invia all'utente una conferma di spedizione dell'articolo. \\
\end{tabular}
\end{minipage} \\

\hline
\textbf{Condizione d'uscita} & L'utente ha acquistato un articolo. \\

\hline
\end{tabular}

\subsubsection{Possibilità per ogni utente di vendere articoli}
\label{UC:4}
\begin{tabular}{|c|p{12cm}|}
\hline
\textbf{ID} & UC4 Vendita \\
\hline
\textbf{Nome} & Vendita di un articolo sulla piattaforma \\
\hline
\textbf{Partecipanti} & Utente registrato \\
\hline
\textbf{Condizione d'ingresso} & Un cliente della piattaforma decide di vendere un articolo. \\
\hline
\textbf{Flusso di eventi} &
\begin{minipage}{12cm}
\begin{tabular}{p{5.5cm} p{5.5cm}}
\textbf{Utente} & \textbf{Sistema} \\
Dopo aver effettuato il login (\ref{UC:1}), fa click sul tasto ``Vendi". \\
& Reindirizza l'utente al form per l'inserimento dei dati dell'articolo da vendere. \\
Inserisce nome, prezzo, descrizione e foto dell'articolo. \\
& Reindirizza l'utente ad una pagina di conferma. \\
Conferma la vendita. \\
\end{tabular}
\end{minipage} \\

\hline
\textbf{Condizione d'uscita} & L'articolo viene messo in vendita sulla piattaforma. \\
\hline
\end{tabular}

\subsubsection{Possibilità di contatto del servizio clienti}
\label{UC:5}
\begin{tabular}{|c|p{12cm}|}
\hline
\textbf{ID} & UC5 Assistenza \\
\hline
\textbf{Nome} & Contatto servizio clienti \\
\hline
\textbf{Partecipanti} & Utente registrato, centralinista \\
\hline
\textbf{Condizione d'ingresso} & Un cliente della piattaforma ha necessità di assistenza. \\
\hline
\textbf{Flusso di eventi} &
\begin{minipage}{12cm}
\begin{tabular}{p{4cm} p{4cm} p{3cm}}
\textbf{Utente} & \textbf{Sistema} & \textbf{Centralinista} \\
Dopo aver effettuato il login (\ref{UC:1}), fa click sul tasto ``Servizio clienti". \\
& Reindirizza l'utente al form per la scrittura di un messaggio per l'assistenza clienti. \\
Scrive il messaggio.  \\
& Inoltra il messaggio ad un centralinista. \\
& & Risponde al messaggio. \\
& Inoltra la risposta. \\
\vdots & \vdots & \vdots \\
Chiude il ticket. \\
\end{tabular}
\end{minipage} \\

\hline
\textbf{Condizione d'uscita} & Il problema dell'utente viene dichiarato risolto e il ticket viene chiuso. \\
\hline
\end{tabular}

\subsubsection{Modifica informazioni profilo}
\label{UC:6}
\begin{tabular}{|c|p{12cm}|}
\hline
\textbf{ID} & UC6 ModificaProfilo \\
\hline
\textbf{Nome} & Modifica informazioni profilo \\
\hline
\textbf{Partecipanti} & Utente registrato \\
\hline
\textbf{Condizione d'ingresso} & Un cliente della piattaforma ha necessità di modificare le informazioni del suo profilo. \\
\hline
\textbf{Flusso di eventi} &
\begin{minipage}{12cm}
\begin{tabular}{p{5.5cm} p{5.5cm}}
\textbf{Utente} & \textbf{Sistema} \\
Dopo aver effettuato il login (\ref{UC:1}), fa click sul tasto ``Profilo". \\
& Reindirizza l'utente alla contenente tutte le informazioni di quel cliente. \\
Fa click su ``Modifica".  \\
& Reindirizza il cliente ad una pagina con un form per la modifica dei dati personali. \\
Inserisce i dati modificati. \\
& Salva le modifiche. \\
\end{tabular}
\end{minipage} \\

\hline
\textbf{Condizione d'uscita} & Le vecchie informazioni vengono sovrascritte. \\
\hline
\end{tabular}

\subsubsection{Recupero password}
\label{UC:7}
\begin{tabular}{|c|p{12cm}|}
\hline
\textbf{ID} & UC7 RecuperoPassword \\
\hline
\textbf{Nome} & Recupero password smarrita \\
\hline
\textbf{Partecipanti} & Utente registrato \\
\hline
\textbf{Condizione d'ingresso} & Un cliente della piattaforma ha dimenticato la sua password e vuole recuperarla. \\
\hline
\textbf{Flusso di eventi} &
\begin{minipage}{12cm}
\begin{tabular}{p{5.5cm} p{5.5cm}}
\textbf{Utente} & \textbf{Sistema} \\
Dalla pagina del login (\ref{UC:1}), fa click su ``Password smarrita?".
& Reindirizza l'utente al form per l'inserimento della propria email. \\
Inserisce la propria email.  \\
& Invia un link per il recupero della password. \\
Apre la casella email e utilizza quel link. \\
&  Reindirizza l'utente ad un form per il ripristino della password. \\
Inserisce la nuova password. \\
& Reindirizza l'utente al form di login, con un messaggio che lo informa della modifica effettuata. \\
\end{tabular}
\end{minipage} \\

\hline
\textbf{Condizione d'uscita} & La password dell'utente viene sovrascritta. \\
\hline
\end{tabular}

\section{Class Diagram}

\section{Statechart Diagram}

\section{Activity Diagram}

\section{Navigational Path}

\section{Mock up}

\end{document}
