\documentclass[12pt,a4paper]{article}
\usepackage[utf8]{inputenc}
\usepackage{amsmath}
\usepackage{amsfonts}
\usepackage{amssymb}
\usepackage{graphicx}
\graphicspath{ {./img/} }
\usepackage{hyperref}
\usepackage{array}
\usepackage[table]{xcolor}
\usepackage{xcolor,colortbl}
\usepackage{multirow}
\usepackage{listings}
\usepackage[a4paper, total={6in, 8in}]{geometry}

\usepackage{titlesec}

\setcounter{secnumdepth}{4}

\titleformat{\paragraph}
{\normalfont\normalsize\bfseries}{\theparagraph}{1em}{}
\titlespacing*{\paragraph}
{0pt}{3.25ex plus 1ex minus .2ex}{1.5ex plus .2ex}

\author{Natale Guadagno, Paolo Patrone}
\title{Object Design Document - TecStore}
\renewcommand{\contentsname}{Contenuti}

\usepackage{hyperref}
\hypersetup{
    colorlinks,
        citecolor=blue,
    filecolor=blue,
    linkcolor=blue,
    urlcolor=blue,
    linktocpage
}

\begin{document}

\maketitle
\newpage
\tableofcontents
\newpage
\newgeometry{top=0.5in,bottom=0.5in,right=0.5in,left=0.5in}
\section*{Partecipanti}
\begin{center}
\begin{tabular} {|c|c|}
\hline
\textbf{Nome} & \textbf{Matricola} \\
\hline
Guadagno Natale & 0512106546 \\
Patrone Paolo & 0512106153 \\
\hline
\end{tabular}
\end{center}
\newpage

\section{Trade-off}
\subsection{Leggibilit\`a del codice e sicurezza vs tempo di sviluppo}
Se \`e vero che nel System Design Document sono state identificate le necessit\`a di avere codice facilmente leggibile ed espandibile, i ristretti tempi di sviluppo impongono una minor attenzione a tale requisito, pur mantenendo un livello minimo. Ci\`o non deve avvenire a discapito della sicurezza del sistema, che deve rispettare i parametri identificati in precedenza.

\subsection{Prestazioni vs Requisiti hardware}
Nonostante i potenziali \emph{drawback} del salvataggio delle immagini all'interno del database MySQL, come l'aumento dei tempi di backup e l'aumento dell'uso di memoria RAM, si è scelto di semplificare l'implementazione e utilizzare tale tecnica.

\subsection{Prestazioni vs Tempi di sviluppo}
Sebbene si sia identificato in precedenza il requisito di poter gestire dinamicamente il carico attraverso più nodi provvisti di \emph{load balancer}, a causa dei tempi ristretti di sviluppo di è deciso di non dare priorità a tale requisito per evitare ulteriori ostacoli e rallentamenti durante lo sviluppo. Rimane un elemento importante che andrebbe aggiunto in una successiva revisione del sistema.


\section{Convenzioni per la stesura del codice}

\subsection{Naming convention}
\`E richiesto utilizzare nomi di variabili, classi e metodi che siano:

\begin{itemize}
\item Descrittivi del loro utilizzo, evitando quanto più possibile ambiguità e nomi poco descrittivi (e.g. ``a", ``abc", e così via).
\item Pronunciabili, evitando il più possibile abbreviazioni ed evitando di utilizzare lingue diverse (e.s. unione italiano ed inglese nello stesso nome)
\item Di lunghezza medio-corta (preferibilmente meno di 12 caratteri), per facilitare la lettura e la scrittura.
\item Usando la pratica del ``camel case" per tutti i nomi composti da pi\`u di una parola, con la prima lettera della prima parola minuscola per la prima parola e maiuscola per tutte le parole successive per i nomi di variabili e metodi, e tutte le parole con la prima lettera maiuscola per tutti i nomi delle classi.
\

\end{itemize}

\subsection{Commenti}
Eventuali commenti devono essere raggruppati all'inizio della funzione per facilitare la lettura della funzione stessa.

\newpage

\subsection{Classi}
I nomi delle classi devono essere immediatamente riconoscibili e riconducibili al loro utilizzo e ogni classe deve essere racchiusa in un file separato. \\

La stesura delle classi deve seguire il seguente ordine:
\begin{enumerate}
\item Dichiarazione del package
\item Import di librerie e classi necessarie
\item Dichiarazione di classe pubblica
\item Dichiarazione di costanti
\item Dichiarazione di variabili di classe
\item Dichiarazione di variabili d'istanza
\item Dichiarazione del costruttore
\item Dichiarazione di metodi pubblici
\item Dichiarazione di Setter e Getter
\item Dichiarazione di metodi privati
\end{enumerate}

\newpage

\section{Packages}
\subsection{Backend}
L'implementazione del sistema è divisa in più \emph{package} per semplificare la distinzione tra le classi:
\begin{itemize}
\item bean
\item control
\item model
\item util
\end{itemize}

\subsubsection{bean}
\begin{center}
\begin{tabular}{|c|c|}
\hline
\rowcolor[HTML]{C0C0C0}
\rowcolor[HTML]{C0C0C0} Classe & Descrizione  \\ \hline
ArticoloBean & \begin{minipage}{10cm} \vspace{5pt}
Rappresentazione di un articolo, con informazioni sul nome, prezzo, quantità disponibile e altre informazioni rilevanti \vspace{5pt}
\end{minipage} \\ \hline
FotoBean & \begin{minipage}{10cm} \vspace{5pt}  Rappresentazione di una foto, contenente oltre alla foto stessa il riferimento all'articolo a cui fa riferimento  \vspace{5pt} \end{minipage}  \\ \hline
MessaggioBean & \begin{minipage}{10cm} \vspace{5pt} Rappresentazione di una risposta ad un ticket, contenente tutte le informazioni rilevanti, come l'autore e il ticket a cui fa riferimento  \vspace{5pt} \end{minipage}  \\ \hline
OrdineBean & \begin{minipage}{10cm} \vspace{5pt} Rappresentazione di un ordine, contenente le informazioni rilevanti come il cliente, la quantità, il codice del \emph{tracking}  \vspace{5pt} \end{minipage}  \\ \hline
TicketBean & \begin{minipage}{10cm} \vspace{5pt} Rappresentazione di un ticket, contenente le informazioni rilevanti \vspace{5pt} \end{minipage}  \\ \hline
UtenteBean & \begin{minipage}{10cm} \vspace{5pt} Rappresentazione di un utente, contenente varie informazioni necessarie  \vspace{5pt} \end{minipage}  \\ \hline
\end{tabular}
\end{center}

\subsubsection{control}
\begin{center}
\begin{tabular}{|c|c|}
\hline
\rowcolor[HTML]{C0C0C0} 
\multicolumn{2}{|c|}{\cellcolor[HTML]{C0C0C0}Carrello} \\ \hline
\rowcolor[HTML]{C0C0C0}  \multicolumn{1}{|c|}{\cellcolor[HTML]{C0C0C0}Classe}  &  Descrizione \\ \hline

AggiornamentoQuantitaCarrelloServlet & \begin{minipage}{10cm} \vspace{5pt}
Servlet adibita all'aggiornamento della quantità di un articolo presente nel carrello \vspace{5pt}
\end{minipage} \\ \hline

AggiuntaAlCarrelloServlet & \begin{minipage}{10cm} \vspace{5pt}
Servlet adibita all'aggiunta di un articolo al carrello \vspace{5pt}
\end{minipage} \\ \hline

GetCarrelloServlet & \begin{minipage}{10cm} \vspace{5pt}
Servlet adibita al recupero dell'elenco degli articoli nel carrello \vspace{5pt}
\end{minipage} \\ \hline

RimozioneDalCarrelloServlet & \begin{minipage}{10cm} \vspace{5pt}
Servlet adibita alla rimozione di un articolo dal carrello \vspace{5pt}
\end{minipage} \\ \hline

\end{tabular}
\end{center}

\begin{center}
\begin{tabular}{|c|c|}
\hline
\rowcolor[HTML]{C0C0C0} 
\multicolumn{2}{|c|}{\cellcolor[HTML]{C0C0C0}Ordine} \\ \hline
\rowcolor[HTML]{C0C0C0}  \multicolumn{1}{|c|}{\cellcolor[HTML]{C0C0C0}Classe}  &  Descrizione \\ \hline

AnnullamentoOrdineServlet & \begin{minipage}{10cm} \vspace{5pt}
Servlet adibita all'annullamento di un ordine \vspace{5pt}
\end{minipage} \\ \hline

ConfermaOrdineServlet & \begin{minipage}{10cm} \vspace{5pt}
Servlet adibita alla conferma della spedizione di un ordine \vspace{5pt}
\end{minipage} \\ \hline

CreazioneOrdineServlet & \begin{minipage}{10cm} \vspace{5pt}
Servlet adibita alla conversione degli articoli presenti nel carrello in ordini \vspace{5pt}
\end{minipage} \\ \hline

DettagliOrdineServlet & \begin{minipage}{10cm} \vspace{5pt}
Servlet adibita al recupero dei dettagli di un ordine \vspace{5pt}
\end{minipage} \\ \hline

ElencoOrdiniMagazziniereServlet & \begin{minipage}{10cm} \vspace{5pt}
Servlet adibita al recupero dell'elenco degli ordini non ancora spediti \vspace{5pt}
\end{minipage} \\ \hline

ElencoOrdiniMagazziniereServlet & \begin{minipage}{10cm} \vspace{5pt}
Servlet adibita al recupero dell'elenco degli ordini per cui è stato richiesto un rimborso \vspace{5pt}
\end{minipage} \\ \hline

RicercaOrdiniClienteServlet & \begin{minipage}{10cm} \vspace{5pt}
Servlet adibita al recupero dello storico ordini di un cliente \vspace{5pt}
\end{minipage} \\ \hline

RimborsoClienteServlet & \begin{minipage}{10cm} \vspace{5pt}
Servlet adibita alla gestione di una richiesta di rimborso effettuata da un cliente \vspace{5pt}
\end{minipage} \\ \hline

RimborsoMgazziniereServlet & \begin{minipage}{10cm} \vspace{5pt}
Servlet adibita alla conferma o al rifiuto di un rimborso da parte di un magazziniere \vspace{5pt}
\end{minipage} \\ \hline

\end{tabular}
\end{center}

\begin{center}
\begin{tabular}{|c|c|}
\hline
\rowcolor[HTML]{C0C0C0} 
\multicolumn{2}{|c|}{\cellcolor[HTML]{C0C0C0}Ticket} \\ \hline
\rowcolor[HTML]{C0C0C0}  \multicolumn{1}{|c|}{\cellcolor[HTML]{C0C0C0}Classe}  &  Descrizione \\ \hline

ChiusuraTicketServlet & \begin{minipage}{10cm} \vspace{5pt}
Servlet adibita alla chiusura di un ticket \vspace{5pt}
\end{minipage} \\ \hline

CreazioneTicketServlet & \begin{minipage}{10cm} \vspace{5pt}
Servlet adibita alla creazione di un ticket \vspace{5pt}
\end{minipage} \\ \hline

DettagliTicketServlet & \begin{minipage}{10cm} \vspace{5pt}
Servlet adibita al recupero dei dettagli di un ticket \vspace{5pt}
\end{minipage} \\ \hline

ElencoTicketCentralinistaServlet & \begin{minipage}{10cm} \vspace{5pt}
Servlet adibita al recupero dell'elenco dei ticket in attesa di risposta \vspace{5pt}
\end{minipage} \\ \hline

ElencoTicketClienteServlet & \begin{minipage}{10cm} \vspace{5pt}
Servlet adibita al recupero dell'elenco dei ticket per un cliente \vspace{5pt}
\end{minipage} \\ \hline

RispostaTicketServlet & \begin{minipage}{10cm} \vspace{5pt}
Servlet adibita alla gestione dei nuovi messaggi relativi ad un ticket \vspace{5pt}
\end{minipage} \\ \hline

\end{tabular}
\end{center}

\begin{center}
\begin{tabular}{|c|c|}
\hline
\rowcolor[HTML]{C0C0C0} 
\multicolumn{2}{|c|}{\cellcolor[HTML]{C0C0C0}Utente} \\ \hline
\rowcolor[HTML]{C0C0C0}  \multicolumn{1}{|c|}{\cellcolor[HTML]{C0C0C0}Classe}  &  Descrizione \\ \hline

AutenticazioneServlet & \begin{minipage}{10cm} \vspace{5pt}
Servlet adibita all'autenticazione di un utente \vspace{5pt}
\end{minipage} \\ \hline

Deautenticazione & \begin{minipage}{10cm} \vspace{5pt}
Servlet adibita alla deautenticazione di un utente \vspace{5pt}
\end{minipage} \\ \hline

DettagliUtenteServlet & \begin{minipage}{10cm} \vspace{5pt}
Servlet adibita al recupero dei dettagli di un utente \vspace{5pt}
\end{minipage} \\ \hline

ModificaUtenteServlet & \begin{minipage}{10cm} \vspace{5pt}
Servlet adibita alla modifica dei dettagli di un utente \vspace{5pt}
\end{minipage} \\ \hline

RegistrazioneUtenteServlet & \begin{minipage}{10cm} \vspace{5pt}
Servlet adibita alla creazione di un nuovo utente \vspace{5pt}
\end{minipage} \\ \hline

RicercaPersonaleServlet & \begin{minipage}{10cm} \vspace{5pt}
Servlet adibita alla ricerca tra i dipendenti della piattaforma \vspace{5pt}
\end{minipage} \\ \hline

RimozioneUtenteServlet & \begin{minipage}{10cm} \vspace{5pt}
Servlet adibita alla rimozione dei dettagli di un utente \vspace{5pt}
\end{minipage} \\ \hline

\end{tabular}
\end{center}

\begin{center}
\begin{tabular}{|c|c|}
\hline
\rowcolor[HTML]{C0C0C0} 
\multicolumn{2}{|c|}{\cellcolor[HTML]{C0C0C0}Vendita} \\ \hline
\rowcolor[HTML]{C0C0C0}  \multicolumn{1}{|c|}{\cellcolor[HTML]{C0C0C0}Classe}  &  Descrizione \\ \hline

AnnullamentoVenditaServlet & \begin{minipage}{10cm} \vspace{5pt}
Servlet adibita all'annullamento di una vendita \vspace{5pt}
\end{minipage} \\ \hline

AutorizzazioneVenditaServlet & \begin{minipage}{10cm} \vspace{5pt}
Servlet adibita all'autorizzazione di una vendita da parte di un centralinista \vspace{5pt}
\end{minipage} \\ \hline

DettagliArticoloServlet & \begin{minipage}{10cm} \vspace{5pt}
Servlet adibita al recupero dei dettagli di un articolo \vspace{5pt}
\end{minipage} \\ \hline

DettagliVenditaServlet & \begin{minipage}{10cm} \vspace{5pt}
Servlet adibita al recupero dei dettagli di una vendita \vspace{5pt}
\end{minipage} \\ \hline

ElencoVenditeCentralinistaServlet & \begin{minipage}{10cm} \vspace{5pt}
Servlet adibita al recupero dell'elenco delle vendite in attesa di autorizzazione \vspace{5pt}
\end{minipage} \\ \hline

ImmaginiServlet & \begin{minipage}{10cm} \vspace{5pt}
Servlet adibita al recupero e all'inserimento delle immagini per gli articoli  \vspace{5pt}
\end{minipage} \\ \hline

InserimentoNuovoArticoloServlet & \begin{minipage}{10cm} \vspace{5pt}
Servlet adibita alla creazione di un nuovo articolo \vspace{5pt}
\end{minipage} \\ \hline

ModificaArticoloServlet & \begin{minipage}{10cm} \vspace{5pt}
Servlet adibita alla modifica dei dati di un articolo \vspace{5pt}
\end{minipage} \\ \hline

RicercaArticoloServlet & \begin{minipage}{10cm} \vspace{5pt}
Servlet adibita alla ricerca di un articolo \vspace{5pt}
\end{minipage} \\ \hline

RicercaVenditaServlet & \begin{minipage}{10cm} \vspace{5pt}
Servlet adibita alla ricerca di un articolo in vendita per un particolare utente \vspace{5pt}
\end{minipage} \\ \hline

RimozioneArticoloServlet & \begin{minipage}{10cm} \vspace{5pt}
Servlet adibita alla rimozione di un articolo \vspace{5pt}
\end{minipage} \\ \hline

\end{tabular}
\end{center}

\subsubsection{Model}

\begin{center}
\begin{tabular}{|c|c|}
\hline
\rowcolor[HTML]{C0C0C0} 
\rowcolor[HTML]{C0C0C0}  \multicolumn{1}{|c|}{\cellcolor[HTML]{C0C0C0}Classe}  &  Descrizione \\ \hline

GestioneAccount & \begin{minipage}{10cm} \vspace{5pt}
Insieme di funzioni relative alla gestione degli account \vspace{5pt}
\end{minipage} \\ \hline

GestioneAssistenza & \begin{minipage}{10cm} \vspace{5pt}
Insieme di funzioni relative alla gestione dei ticket e dei messaggi \vspace{5pt}
\end{minipage} \\ \hline

GestioneCarrello & \begin{minipage}{10cm} \vspace{5pt}
Insieme di funzioni relative alla gestione del carrello \vspace{5pt}
\end{minipage} \\ \hline

GestioneOrdine & \begin{minipage}{10cm} \vspace{5pt}
Insieme di funzioni relative alla gestione degli ordini \vspace{5pt}
\end{minipage} \\ \hline

GestioneOrdine & \begin{minipage}{10cm} \vspace{5pt}
Insieme di funzioni relative alla gestione degli articoli e delle vendite \vspace{5pt}
\end{minipage} \\ \hline

\end{tabular}
\end{center}
\newpage 

\subsubsection{Metodi delle classi}


Tutte le servlet descritte rispondono unicamente a richieste di tipo POST, ignorando totalmente le richieste di tipo GET, fatta eccezione per la servlet ``ImmaginiServlet", che risponde con il file dell'immagine richiesta (se esiste) per le richieste GET e gestisce l'inserimento per le richieste POST.

Pertanto, l'interfaccia di tutte le classi Servlet è del tipo

\begin{lstlisting}[language=Java]
package control.<>;

import ...;

@WebServlet("/<nomeServlet>")

public class <NomeClasse> extends HttpServlet{
	protected void doGet(HttpServletRequest request, 
					HttpServletResponse response)
			throws ServletException, IOException 
		{
			...
			...
			...
		}
	
	protected void doPost(HttpServletRequest request,
					HttpServletResponse response)
			throws ServletException, IOException 
		{
			...
			...
			...
		}
}
\end{lstlisting}

\newpage

Ogni model implementa più funzioni e, in alcuni casi, più versioni della stessa funzione, con diversi parametri di input o altre modifiche di lieve entità. Per brevità, sarà riportata solo la versione principale di ciascuna di queste funzioni.

\begin{lstlisting}[language=Java]
package model;

import ...;

public class GestioneAccount {
	// Dato un codice fiscale, 
	// restituisce true se esiste nel database
	public boolean exists(String) { }
	
	// Dati email e password,
	// restituisce true se esistono nel database e coincidono
	public boolean autenticazione(String, String) { }
	
	// Data una stringa, 
	// ne restituisce la sua versione cifrata
	public String encryptString(String) { }
	
	// Data una stringa, 
	// ne restituisce la sua versione decifrata
	public String decryptString(String) { }
	
	// Dato un codice fiscale,
	// ne restituisce l'UtenteBean completo
	public UtenteBean dettagliUtente(String) { }
	
	// Dato il codice fiscale di chi sta compiendo l'operazione
	// e il codice fiscale dell'utente da eliminare,
	// restituisce true se l'eliminazione va a buon fine
	public boolean eliminaUtente(String, String) { }
	
	// Data una lunghezza,
	// restituisce una stringa alfanumerica pseudocasuale
	public String generatePassword(int) { }
	
	// Dato un codice fiscale,
	// restituisce la tipologia utente
	public int getTipologia(String) { }
	
	// Dato il codice fiscale di chi sta compiendo l'operazione
	// e i nuovi dettagli,
	// restituisce true se la modifica dell'utente va a buon fine	
	public boolean modificaUtente(String, UtenteBean) { }
	
	// Dati i dettagli dell'utnete da registrare,
	// restituisce true se la registrazione dell'utente va a buon fine
	public boolean registrazioneUtente(UtenteBean) { }
	

	
	
	
	// Dato il nome, cognome o il codice fiscale di un dipendente,
	// restituisce un elenco di dipendenti corrispondenti
	public ArrayList<UtenteBean> ricercaDipendenti(String) { }
}
\end{lstlisting}

\begin{lstlisting}[language=Java]
package model;

import ...;

public class GestioneAssistenza {
	// Dato l'ID di un ticket e il nuovo stato, 
	// restituisce true se il cambiamento di stato va a buon fine
	public boolean cambiaStato(String, String) { }
	
	// Dato il codice fiscale del creatore, la tipologia di ticket
	// e il primo messaggio
	// restituisce true se la creazione del ticket va a buon fine
	public boolean creazioneTicket(String, String, String) { }
	
	// Dato l'ID di un ticket, 
	// ne restituisce i dettagli
	public TicketBean dettagliTicket(String) { }
	
	// Dato l'ID di un ticket, 
	// ne restituisce i messaggi associati
	public ArrayList<MessaggioBean> elencoMessaggiTicket(String) { }
	
	// Dato il limite di risultati da ottenere,
	// restituisce l'elenco dei ticket in attesa di risposta
	public ArrayList<TicketBean> elencoTicketCentralinista(int) { }
	
	// Dato un codice fiscale e il limite di risultati da ottenere,
	// restituisce l'elenco dei ticket di un cliente
	public ArrayList<TicketBean> elencoTicketCliente(String, int) { }
	
	// Dato l'ID di un ticket,
	// il codice fiscale di chi scrive e un messaggio
	// restituisce true se l'inserimento del messaggio va a buon fine
	public boolean rispostaTicket(String, String, String) { }

}
\end{lstlisting}

\newpage

\begin{lstlisting}[language=Java]
package model;

import ...;

public class GestioneCarrello {
	// Dati il codice fiscale dell'utente,
	// l'ID dell'articolo e la nuova quantita',
	// restituisce true se l'aggiornamento del carrello va a buon fine
	public boolean aggiornamentoQuantita(String, String, int) { }
	
	// Dati il codice fiscale dell'utente,
	// l'ID dell'articolo e la nuova quantita',
	// restituisce true se l'aggiunta al carrello va a buon fine
	public boolean aggiuntaArticolo(String, String, int) { }
	
	// Dato il codice fiscale dell'utente,
	// restituisce l'elenco degli articoli nel carrello
	public static ArrayList<ArticoloBean> GetCarrello(String) { }
	
	
	// Dati il codice fiscale dell'utente e
	// l'ID dell'articolo 
	// restituisce true se la rimozione va a buon fine
	public boolean rimozioneArticolo(String, String)  { }
}
\end{lstlisting}

\newpage

\begin{lstlisting}[language=Java]
package model;

import ...;

public class GestioneOrdine {
	// Dato l'ID di un ordine e il nuovo stato, 
	// restituisce true se il cambiamento di stato va a buon fine
	public boolean cambiaStato(String, String) { }

	// Dato il codice fiscale di un utente,
	// restituisce true se gli ordini vengono creati correttamente
	public boolean creazioneOrdine(String) { }
	
	// Dato l'ID di un ordine,
	// ne restituisce i dettagli
	public OrdineBean dettagliOrdine(String) { }
	
	// Restituisce l'elenco degli ordini in attesa di spedizione
	public ArrayList<OrdineBean> elencoOrdiniMagazziniere() { }
	
	// Dato il limite di risultati da ottenere,
	// restituisce l'elenco degli ordini in attesa di rimborso
	public ArrayList<OrdineBean> elencoRimborsi(int) { }
	
	// Dati il codice fiscale del cliente,
	// il nome dell'articolo interessato e 
	// il limite di risultati da ottenere,
	// restituisce l'elenco degli articoli ordinati corrispondenti
	public ArrayList<OrdineBean> ricercaOrdiniCliente
					(String, String, int) { }
	
	// Dati l'ID dell'ordine e il codice del tracking,
	// restituisce true se l'inserimento va a buon fine
	public boolean setCodiceTracciamento(String, String) { }
}
\end{lstlisting}

\newpage

\begin{lstlisting}[language=Java]
package model;

import ...;

public class GestioneVendita {
	// Dato l'ID di un articolo e il nuovo stato, 
	// restituisce true se il cambiamento di stato va a buon fine
	public boolean cambiaStato(String, String) { }
	
	// Dato l'ID di un articolo,
	// ne restituisce i dettagli
	public ArticoloBean dettagliArticolo(String) { }
	
	// Dato il limite di risultati da ottenere,
	// restituisce l'elenco degli articoli in attesa di approvazione
	public ArrayList<ArticoloBean> elencoVenditeCentralinista(int) { }
	
	// Dato l'ID di una foto,
	// restituisce la foto stessa
	public byte[] getFoto(String) { }
	
	// Dato l'ID di un articolo e una foto,
	// restituisce true se l'inserimento va a buon fine
	public boolean inserimentoFoto(String, InputStream) { }
	
	// Dati i dettagli di un nuovo articolo,
	// restituisce l'ID del nuovo articolo se l'inserimento va a buon fine
	public String inserimentoNuovoArticolo(ArticoloBean) { }
	
	// Dati i dettagli dell'articolo,
	// restituisce true se la modifica va a buon fine
	public String modificaArticolo(ArticoloBean) { }
	
	// Data una parola chiave,
	// restituisce l'elenco degli articoli in vendita corrispondenti
	public ArrayList<ArticoloBean> ricercaArticolo(String) { }
	
	// Dato il codice fiscale di chi esegue l'operazione e
	// l'ID dell'articolo,
	// restituisce true se la rimozione va a buon fine
	public boolean rimozioneArticolo(String, String) { }
	
	// Dato l'ID dell'articolo,
	// restituisce true se la rimozione delle foto va a buon fine
	public boolean rimozioneFoto(String) { }
}
\end{lstlisting}


\newpage

\subsection{Frontend}
Tutte le pagine sono racchiuse nel package di default, ``webapp".

\begin{center}
\begin{tabular}{|c|c|}
\hline
\rowcolor[HTML]{C0C0C0} 
\rowcolor[HTML]{C0C0C0}  \multicolumn{1}{|c|}{\cellcolor[HTML]{C0C0C0}File}  &  Descrizione \\ \hline

areaVenditori & \begin{minipage}{10cm} \vspace{5pt}
Pagina iniziale per un cliente che vuole vendere un articolo o visualizzare i dettagli di un articolo messo in vendita in precedenza \vspace{5pt}
\end{minipage} \\ \hline

autenticazione & \begin{minipage}{10cm} \vspace{5pt}
Modulo per l'autenticazione di un utente \vspace{5pt}
\end{minipage} \\ \hline

carrello & \begin{minipage}{10cm} \vspace{5pt}
Pagina per la visualizzazione del contenuto del carrello \vspace{5pt}
\end{minipage} \\ \hline

centroassistenza & \begin{minipage}{10cm} \vspace{5pt}
Pagina per la visualizzazione dell'elenco dei ticket aperti \vspace{5pt}
\end{minipage} \\ \hline

creazioneTicket & \begin{minipage}{10cm} \vspace{5pt}
Pagina per la creazione di un nuovo ticket \vspace{5pt}
\end{minipage} \\ \hline

dettagliArticolo & \begin{minipage}{10cm} \vspace{5pt}
Pagina per la visualizzazione dei dettagli di un articolo \vspace{5pt}
\end{minipage} \\ \hline

dettagliOrdine & \begin{minipage}{10cm} \vspace{5pt}
Pagina per la visualizzazione dei dettagli di un ordine \vspace{5pt}
\end{minipage} \\ \hline

dettagliutente & \begin{minipage}{10cm} \vspace{5pt}
Pagina per la visualizzazione dei dettagli di un utente, con possibilità di modifica \vspace{5pt}
\end{minipage} \\ \hline

elencoOrdiniMagazziniere & \begin{minipage}{10cm} \vspace{5pt}
Pagina per la visualizzazione dell'elenco degli ordini in attesa di spedizione \vspace{5pt}
\end{minipage} \\ \hline

elencoRimborsiMagazziniere & \begin{minipage}{10cm} \vspace{5pt}
Pagina per la visualizzazione dell'elenco dei rimborsi in attesa di conferma \vspace{5pt}
\end{minipage} \\ \hline

elencoVenditeCentralinista & \begin{minipage}{10cm} \vspace{5pt}
Pagina per la visualizzazione dell'elenco delle vendite in attesa di approvazione \vspace{5pt}
\end{minipage} \\ \hline

errore & \begin{minipage}{10cm} \vspace{5pt}
Pagina per la visualizzazione di un messaggio di errore \vspace{5pt}
\end{minipage} \\ \hline

gestioneAssistenza & \begin{minipage}{10cm} \vspace{5pt}
Pagina per la visualizzazione dell'elenco dei ticket in attesa di risposta\vspace{5pt}
\end{minipage} \\ \hline

gestionepersonale & \begin{minipage}{10cm} \vspace{5pt}
Pagina iniziale per un amministratore del personale, da cui ricercare un dipendente \vspace{5pt}
\end{minipage} \\ \hline

index & \begin{minipage}{10cm} \vspace{5pt}
Pagina iniziale per un cliente, da cui può accedere a tutte le funzioni \vspace{5pt}
\end{minipage} \\ \hline

inserimentoCartaDiCredito & \begin{minipage}{10cm} \vspace{5pt}
Modulo per l'inserimento dei dati della carta di credito da utilizzare per pagare gli ordini \vspace{5pt}
\end{minipage} \\ \hline

inserimentoImmagini & \begin{minipage}{10cm} \vspace{5pt}
Modulo per l'inserimento di foto dopo la creazione di una vendita \vspace{5pt}
\end{minipage} \\ \hline

modificaArticolo & \begin{minipage}{10cm} \vspace{5pt}
Modulo per la modifica di un articolo in vendita \vspace{5pt}
\end{minipage} \\ \hline

modificaPassword & \begin{minipage}{10cm} \vspace{5pt}
Modulo per la modifica della password di un cliente \vspace{5pt}
\end{minipage} \\ \hline

nuovaVendita & \begin{minipage}{10cm} \vspace{5pt}
Modulo per la creazione di una nuova vendita \vspace{5pt}
\end{minipage} \\ \hline

paginainiziale & \begin{minipage}{10cm} \vspace{5pt}
Pagina iniziale per i dipendenti della piattaforma \vspace{5pt}
\end{minipage} \\ \hline

registrazione & \begin{minipage}{10cm} \vspace{5pt}
Modulo per la registrazione di un utente \vspace{5pt}
\end{minipage} \\ \hline

rispostaTicket & \begin{minipage}{10cm} \vspace{5pt}
Modulo per l'inserimento di una risposta ad un ticket \vspace{5pt}
\end{minipage} \\ \hline

risultatiRicerca & \begin{minipage}{10cm} \vspace{5pt}
Pagina per la visualizzazione dei risultati di una ricerca \vspace{5pt}
\end{minipage} \\ \hline

storicoOrdini & \begin{minipage}{10cm} \vspace{5pt}
Pagina per la visualizzazione dello storico degli ordini per un cliente \vspace{5pt}
\end{minipage} \\ \hline

successo & \begin{minipage}{10cm} \vspace{5pt}
Pagina per la visualizzazione dei messaggi di conferma per alcune operazioni \vspace{5pt}
\end{minipage} \\ \hline

\end{tabular}
\end{center}

\end{document}
